\documentclass{amsart}
\usepackage{amsmath,amssymb,amsthm}

\newtheorem{proposition}{Proposition}

\title{Nice Proofs}
\author{Scallop Ye}

\begin{document}

\maketitle

\begin{proposition}
    Let $f:\mathbb{R}\to\mathbb{R}$ be a continuous function with period $T$, and let $x_0\in\mathbb{R}$.
    If $f(x_0)\neq0$ and $\int_{0}^{T}f(x)dx=0$, then $f$ has
    at least two zeros in the interval $I=(x_0,x_0+T)$.
\end{proposition}

\begin{proof}
    The function $f$ is periodic with period $T$, so
    \begin{align*}
        f(x_0+T)                 & =f(x_0)\neq0,          \\
        \int_{x_0}^{x_0+T}f(x)dx & =\int_{0}^{T}f(x)dx=0.
    \end{align*}
    Without loss of generality we may assume that $f(x_0+T)=f(x_0)>0$.
    By the mean value theorem for definite integrals, there exists a $c\in I$ such that
    \[f(c)=\frac{1}{T}\int_{x_0}^{x_0+T}f(x)dx=0.\]
    Hence $f$ has at least one zero in $I$. Suppose for sake of contradiction that
    $f$ has only one zero in $I$. Again, suppose that
    $f(x)<0$ for some $x\in I$. But by the intermediate value theorem
    $f$ has an extra zero in $(x_0,x)$ or $(x,x_0+T)$, a contradiction. Thus we have $f(x)\ge0$ for all $x\in I$.
    Since $f$ is continuous, there exists an $a\in I$ such that
    \[f(x)\ge\frac{f(x_0)}{2}\quad\forall x\in(x_0,a).\]
    By the properties of the Riemann integral
    \begin{align*}
        \int_{x_0}^{a}f(x)dx\ge\int_{x_0}^{a}\frac{f(x_0)}{2}dx=\frac{a-x_0}{2}f(x_0)>0,\\
        \int_{a}^{x_0+T}f(x)dx\ge0,\\
        \int_{x_0}^{x_0+T}f(x)dx=\int_{x_0}^{a}f(x)dx+\int_{a}^{x_0+T}f(x)dx>0.
    \end{align*}
    This contradicts the fact that $\int_{x_0}^{x_0+T}f(x)dx=0$.
    Thus $f$ has at least two zeros in $I$, as desired.
\end{proof}

\end{document}