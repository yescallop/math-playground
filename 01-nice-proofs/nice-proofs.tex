\documentclass{amsart}
\usepackage{amsmath,amssymb,amsthm}

\newtheorem{proposition}{Proposition}

\title{Nice Proofs}
\author{Scallop Ye}

\begin{document}

\maketitle

\begin{proposition}
    Let $f$ be a continuous function with period $T$, and let $x_0$ be a point.
    If $f(x_0)\neq0$ and $\int_{0}^{T}f(x)dx=0$, then $f$ has
    at least two zeros in the interval $I=(x_0,x_0+T)$.
\end{proposition}

\begin{proof}
    The function $f$ is periodic with period $T$, so
    \begin{align*}
        f(x_0+T)                 & =f(x_0)\neq0,          \\
        \int_{x_0}^{x_0+T}f(x)dx & =\int_{0}^{T}f(x)dx=0.
    \end{align*}
    Without loss of generality we may assume that $f(x_0+T)=f(x_0)>0$.
    By the mean value theorem for definite integrals, there exists $c\in I$ such that
    \[f(c)=\frac{1}{T}\int_{x_0}^{x_0+T}f(x)dx=0.\]
    Hence $f$ has at least one zero in $I$. Suppose for sake of contradiction that
    $f$ has only one zero in $I$. Again, suppose for sake of contradiction that
    there exists $m\in I$ such that $f(m)<0$. So by the intermediate value theorem
    $f$ has an extra zero in $(x_0,m)$ or $(m,x_0+T)$, a contradiction. Thus we have
    \[f(x)\ge0\quad\forall x\in I,\]
    and as $f$ has only one zero in $I$
    \[f(x)>0 \quad\forall x\in I-\{c\}.\]
    By the mean value theorem for definite integrals, there exists $x_1\in(x_0,c)$ and
    $x_2\in(c,x_0+T)$ such that
    \begin{align*}
        0<f(x_1) & =\frac{1}{c-x_0}\int_{x_0}^{c}f(x)dx,     \\
        0<f(x_2) & =\frac{1}{x_0+T-c}\int_{c}^{x_0+T}f(x)dx.
    \end{align*}
    Then we have
    \[\int_{x_0}^{x_0+T}f(x)dx=\int_{x_0}^{c}f(x)dx+\int_{c}^{x_0+T}f(x)dx>0,\]
    contradicting the fact that $\int_{x_0}^{x_0+T}f(x)dx=0$. This contradiction gives the proof.
\end{proof}

\end{document}