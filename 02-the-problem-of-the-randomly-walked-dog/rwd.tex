\documentclass{amsart}
\usepackage{amsmath,amssymb,amsthm,framed}

\DeclareMathOperator{\walk}{walk}
\DeclareMathOperator{\simp}{simp}

\renewcommand{\phi}{\varphi}

\newtheorem{proposition}{Proposition}[section]
\newtheorem{conjecture}[proposition]{Conjecture}

\theoremstyle{definition}
\newtheorem{definition}[proposition]{Definition}

\newtheoremstyle{problem}{}{}{}{}
{\bfseries}{} % Note that final punctuation is omitted.
{\newline}{}

\theoremstyle{problem}
\newtheorem{problem}[proposition]{Problem}

\theoremstyle{remark}
\newtheorem*{remark}{Remark}

\title{The Problem of the Randomly Walked Dog}
\author{Scallop Ye}

\begin{document}

\begin{abstract}
    In 1905, Karl Pearson \cite{pearson} proposed on \emph{Nature} the
    famous problem of the random walk. In this article, we present a new
    combinatorial optimization problem in the spirit of Pearson's one,
    along with some preliminary results based on computer simulation.
\end{abstract}

\maketitle

\section{Introduction}

A man walks the dog every day. The dog will prepare $n$ cards beforehand ($n\ge3$),
each containing a direction and a distance.
They start from home and repeat the following process $n$ times:

\begin{enumerate}
    \item choose a card randomly;
    \item turn to the direction given on the card;
    \item walk the distance given on the card in a straight line;
    \item throw the card away.
\end{enumerate}

After these $n$ rounds, they shall be exactly back home. The dog wins a treat
if they \emph{never} visited any place more than once except their home.

Now, the dog asks for your help in maximizing the probability of winning the
treat in a walk. Note that too much calculation exhausts the dog.

\begin{remark}
    As one might have seen, this problem differs from Pearson’s one in that
    it is not inherently probabilistic. We will be using permutations
    instead to formally define the problem in the next section.
\end{remark}

\section{Definitions}

Let us take a look at the cards first. The cards the dog prepares should represent
a sequence of nonzero vectors that sum to zero in $\mathbb{R}^2$, or in $\mathbb{Z}^2$
without loss of generality. We may then define

\begin{definition}
    A \emph{step vector} is a nonzero vector in $\mathbb{Z}^2$.
    An \emph{$n$-step sequence} $S$, or generally a \emph{step sequence},
    is a finite sequence $(s_1,s_2,\dots,s_n)$ of step vectors;
    we say that $S$ is \emph{zero-sum} iff $\sum_{i=1}^{n}s_i=(0,0)$.
\end{definition}

Then in order to study the points the pair visits in a walk, we shall introduce
polygonal paths:

\begin{definition}
    A \emph{polygonal path} $P$ is a finite sequence $(p_0,p_1,\dots,p_n)$
    of points called its \emph{vertices}; the line segments
    $\overline{p_0p_1},\overline{p_1p_2},\dots,\overline{p_{n-1}p_n}$
    are called its \emph{edges}.
\end{definition}

The winning conditions for the dog naturally translate to whether the polygonal
path formed in a walk does not ``intersect itself'', or in other words whether
the path is simple.

\begin{definition}
    A polygonal path $(p_0,p_1,\dots,p_n)$ is \emph{simple}
    iff for all $0<i\le j<n$,
    \[
        \overline{p_{i-1}p_i}\cap\overline{p_jp_{j+1}}=
        \begin{cases}
            \{p_i\}            & \text{if $i=j$},                     \\
            \{p_0\}\cap\{p_n\} & \text{if $n\ge3$ and $(i,j)=(1,n-1)$}, \\
            \varnothing        & \text{otherwise}.
        \end{cases}
    \]
\end{definition}

Geometrically, a simple path is one in which only consecutive
edges intersect and only at their endpoints; it may also
be \emph{closed} as specially dealt with in the second case above.
Then, we define a way to create a path from a step sequence:

\begin{definition}
    Let $S=(s_1,s_2,\dots,s_n)$ be a step sequence. The \emph{walk} of $S$
    is the polygonal path $(p_0,p_1,\dots,p_n)$
    where $p_0=(0,0)$ and $p_i=p_{i-1}+s_i$ for all $1\le i\le n$.
\end{definition}

Finally, we define the value we are maximizing and then the problem:

\begin{definition}
    The \emph{simplicity} of a step sequence $S$
    is the number of permutations $S'$ of $S$
    such that the walk of $S'$ is simple.
\end{definition}

\begin{problem}[\textsc{The Problem of the Randomly Walked Dog}]
\label{problem:rwd}
\begin{framed}
    \begin{tabular}{@{}ll}
        \textit{Instance:} &
        A natural number $n\ge3$. \\

        \textit{Task:}     &
        Find a zero-sum $n$-step sequence of maximum simplicity.
    \end{tabular}
    \vskip -2pt
\end{framed}
\end{problem}

\section{Preliminary Results}

Let us fix $n\ge3$. We first present a proposition which
implies that it is always possible for the dog to win.


\begin{proposition}
    Let $S$ be a zero-sum $n$-step sequence. If the step vectors in $S$ are not all collinear,
    then there exists a permutation $S'$ of $S$ such that the walk of $S'$ is simple.
\end{proposition}

\begin{proof}
    \emph{Warning: This is only a sketch of the proof with nontrivial gaps.}

    Let $S'=(s_1,s_2,\dots,s_n)$ be a permutation of $S$ such that, with
    $s_i=\langle r_i,\angle\theta_i\rangle$ and $-\pi<\theta_i\le\pi$ for all $1\le i\le n$,
    the sequence $(\theta_1,\theta_2,\dots,\theta_n)$ is increasing.
    Let $P$ be the walk of $S'$. Since $S$ is zero-sum, $S'$ is also zero-sum and thus $P$
    is a polygon. One can show that every interior angle of $P$ is less than
    or equal to $\pi$, and so $P$ is convex.
    A convex polygon whose edges are not all collinear is simple.
\end{proof}

By computer simulation on Problem \ref{problem:rwd}, we observe that the maximum
simplicity on record follows a very simple formula, hence the following conjecture.

\begin{conjecture}
    The maximum simplicity of a zero-sum $n$-step sequence
    is exactly $2\cdot n!/(n-1)$.
\end{conjecture}

A promising start indeed. This makes me think that a simple solution ought to be found,
if only in a form that I can at all understand.

\begin{thebibliography}{9}

    \bibitem{pearson}
    Pearson, K.
    The Problem of the Random Walk.
    \emph{Nature}
    \textbf{72},
    294 (1905).

\end{thebibliography}

\end{document}