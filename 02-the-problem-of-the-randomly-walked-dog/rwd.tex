\documentclass{amsart}
\usepackage{amsmath,amssymb,amsthm,framed}

\DeclareMathOperator{\walk}{walk}
\DeclareMathOperator{\simp}{simp}

\newcommand{\olsi}[1]{\,\overline{\!{#1}}} % overline short italic

\newtheorem{proposition}{Proposition}[section]
\newtheorem{conjecture}[proposition]{Conjecture}

\theoremstyle{definition}
\newtheorem{definition}[proposition]{Definition}

\newtheoremstyle{problem}{}{}{}{}
{\bfseries}{} % Note that final punctuation is omitted.
{\newline}{}

\theoremstyle{problem}
\newtheorem{problem}[proposition]{Problem}

\theoremstyle{remark}
\newtheorem*{remark}{Remark}

\title{The Problem of the Randomly Walked Dog}
\author{Scallop Ye}

\begin{document}

\begin{abstract}
    In 1905, Karl Pearson \cite{pearson} proposed on \emph{Nature} the
    famous problem of the random walk. In this article, we present a new
    combinatorial optimization problem in the spirit of Pearson's problem.
\end{abstract}

\maketitle

\section{Introduction}

A man walks the dog every day. The dog will prepare $n$ cards beforehand.
Each card contains a positive distance and a direction that is neither identical
nor opposite to that on another card. Once the cards are ready, the walk begins.

They start from a point $O$ and repeat the following process $n$ times:

\begin{enumerate}
    \item choose a card randomly;
    \item turn to the direction given on the card;
    \item walk the distance given on the card in a straight line;
    \item throw the card away.
\end{enumerate}

After these $n$ rounds, the walk ends and the dog wins a treat if they visited
any point \emph{twice or more} in the walk, except if the \emph{only} point
they visited twice or more is $O$ and it was \emph{only} visited once at the
beginning and once at the end of the walk.

Now, the dog asks for your help in maximizing the probability of winning the
treat in a walk. The man is known to sometimes restrict the value of $n$ and
sometimes not. Note that too much calculation exhausts the dog.

\begin{remark}
    As one might have seen, this problem differs from Pearson’s one in that
    it is not inherently probabilistic. We will be using permutations
    instead to formally define the problem in the next section.
\end{remark}

\section{Definitions}

Let us take a look at the cards first. The cards the dog prepares should represent
a sequence of noncollinear vectors in $\mathbb{R}^2$, or in $\mathbb{Z}^2$
without loss of generality. We may call these vectors \emph{steps} and define

\begin{definition}
    An \emph{$n$-step sequence} $S$, or generally a step sequence,
    is a finite sequence $(s_1,s_2,\dots,s_n)$
    of noncollinear vectors in $\mathbb{Z}^2$.
\end{definition}

Then in order to study the points the pair visits in a walk, we shall introduce
polygonal paths:

\begin{definition}
    A \emph{polygonal path} $P$ is a finite sequence $(p_1,p_2,\dots,p_n)$
    of points called its \emph{vertices}; the line segments
    $\overline{p_1p_2},\overline{p_2p_3},\dots,\overline{p_{n-1}p_n}$
    are called its \emph{edges}.
\end{definition}

The winning conditions for the dog naturally translate to whether the polygonal
path formed in a walk does not ``intersect itself'', or in other words whether
the path is simple.

\begin{definition}
    A polygonal path $(p_1,p_2,\dots,p_n)$ is \emph{simple}
    iff for all $1<i\le j<n$,
    \[
        \overline{p_{i-1}p_i}\cap\overline{p_jp_{j+1}}=
        \begin{cases}
            \{p_i\}            & \text{if $i=j$},                     \\
            \{p_1\}\cap\{p_n\} & \text{if $n>3$ and $(i,j)=(2,n-1)$}, \\
            \varnothing        & \text{otherwise}.
        \end{cases}
    \]
\end{definition}

Note that a simple polygonal path can be $closed$, as specially dealt with
in the second case above. We also want the same to be the case for our problem,
because if it were not the dog would be so bored as all it has to do is close the path!

Then, we would like to form a polygonal path by \emph{walking} the steps:

\begin{definition}
    Let $S=(s_1,s_2,\dots,s_n)$ be a step sequence. The \emph{walk} of $S$, denoted
    $\walk(S)$, is the polygonal path $(p_0,p_1,\dots,p_n)$
    where $p_0=(0,0)$ and $p_i=p_{i-1}+s_i$ for all $1\le i\le n$.
\end{definition}

Finally, we define the value we are optimizing and then the problem:

\begin{definition}
    Let $S$ be a step sequence. The \emph{simplicity} of $S$,
    denoted $\simp(S)$, is the number of permutations $S'$ of $S$
    such that $\walk(S')$ is simple.
\end{definition}

\begin{problem}[\textsc{The Problem of the Randomly Walked Dog}]
\begin{framed}
    \begin{tabular}{@{}ll}
        \textit{Instance:} &
        A natural number $n\ge2$. \\

        \textit{Task:}     &
        Find an $n$-step sequence of minimum simplicity.
    \end{tabular}
    \vskip -2pt
\end{framed}
\end{problem}

\begin{thebibliography}{9}

    \bibitem{pearson}
    Pearson, K.
    The Problem of the Random Walk.
    \emph{Nature}
    \textbf{72},
    294 (1905).

\end{thebibliography}

\end{document}