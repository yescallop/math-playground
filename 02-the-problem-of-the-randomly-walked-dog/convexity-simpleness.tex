\documentclass{amsart}
\usepackage{amsmath,amssymb,amsthm}
\usepackage{xeCJK}
\setCJKmainfont{SimHei}

\newtheorem{proposition}{Proposition}[section]
\newtheorem{lemma}[proposition]{Lemma}
\newtheorem{theorem}[proposition]{Theorem}

\theoremstyle{definition}
\newtheorem{definition}[proposition]{Definition}

\title{When is a Convex Closed Curve not Simple?}
\author{Scallop Ye}

\begin{document}

\begin{abstract}
    In this article, we provide analytical definitions of closed curves
    and of simpleness and convexity of closed curves. We will show
    that a convex closed curve is simple if and only if
    it does not lie entirely on a line.
\end{abstract}

\maketitle

\section{Definitions}

We first follow Milnor \cite{milnor} to define the concepts in a more rigorous manner.

\begin{definition}[Closed curves]
    A \emph{closed curve} in Euclidean $n$-space $\mathbb{E}^n$ is a
    continuous, periodic function $\gamma:\mathbb{R}\to\mathbb{R}^n$
    which is not constant in any interval.
\end{definition}

One can show that a closed curve has a least positive period.
For convenience we redefine the notion of period as follows.

\begin{definition}[Periods]
    Let $\gamma$ be a closed curve of least positive period $p$,
    and let $x\in\mathbb{R}$. We call $p$ \emph{the period} of $\gamma$,
    and the interval $P=[x,x+p)$ \emph{a period} of $\gamma$.
    We say that a closed curve $\gamma'$ is \emph{of period} $p'$
    iff the period of $\gamma'$ is $p'$.
\end{definition}

Here we restricted periods to be right-open, since every
definition in this article that relies on right-open periods can be
proved equivalent to a left-open version.

\begin{definition}[Simpleness]
    Let $\gamma$ be a closed curve of period $p$.
    We say that $\gamma$ is \emph{simple}
    iff for all $t_1,t_2\in\mathbb{R}$, $\gamma(t_1)=\gamma(t_2)$
    only when $(t_1-t_2)/p$ is an integer.
\end{definition}

\begin{definition}[Convexity]
    Let $\gamma$ be a closed curve in $\mathbb{E}^2$.
    We say that $\gamma$ is \emph{convex} iff
    for every line $L\subset\mathbb{R}^2$,
    there exists a period $P$ of $\gamma$ such that
    the set $\{t\in P:\gamma(t)\in L\}$ either
    has cardinality $\le2$ or is a nontrivial interval.
\end{definition}

\begin{proposition}
    A closed curve $\gamma$ in $\mathbb{E}^2$ is convex iff
    for every $u\in\mathbb{S}^1$ and $y\in\mathbb{R}$,
    there exists a period $P$ of $\gamma$ such that
    the set $\{t\in P:u\cdot\gamma(t)=y\}$ either
    has cardinality $\le2$ or is a nontrivial interval.
\end{proposition}

\begin{proof}
    It is clear that every line $L\subset\mathbb{R}^2$
    can be specified by a normal vector $u\in\mathbb{S}^1$
    and a distance $y\in\mathbb{R}$ from the origin such that
    $L=\{x\in\mathbb{R}^2:u\cdot x=y\}$.
\end{proof}

\begin{definition}
    Let $\gamma$ be a closed curve in $\mathbb{E}^n$,
    let $u\in\mathbb{S}^{n-1}$,
    and let $P$ be any period of $\gamma$.
    Define $\mu(\gamma,u)$ to be the cardinality of the set
    $\{t\in P:\text{the function }u\cdot\gamma\text{ attains a local maximum at }t\}$
    if it is finite, or $\infty$ otherwise.
\end{definition}

The $\mu(\gamma,u)$ defined above is clearly unique as with
many other properties of a periodic function over a period.
We then prove a generalization of the necessity part
of \cite[Lemma 3.3]{milnor} in the hope that
our definitions are equivalent to Milnor's.

\begin{proposition}
    Let $\gamma$ be a closed curve in $\mathbb{E}^2$.
    If $\gamma$ is convex, then for every $u\in\mathbb{S}^1$
    either $\mu(\gamma,u)=1$ or $\mu(\gamma,u)=\infty$.
\end{proposition}

\begin{proof}
    Suppose that $2\le\mu(\gamma,u)<\infty$ for some $u\in\mathbb{S}^1$.
    Let $f:\mathbb{R}\to\mathbb{R}$ be the function $u\cdot\gamma$,
    let $P=[x,x+p)$ be any period of $\gamma$, and let $t_1<t_2\in P$
    such that $f$ attains local maxima at $t_1$ and $t_2$.
    Suppose that $f(t_1)\ge f(t_2)$.
    Since $f$ attains a finite number of local maxima in $P$,
    there exists $a,b,y$ with $t_1<a<t_2<b<x+p$ such that
    $f(a),f(b)<y<f(t_2)\le f(t_1)$.
    By the intermediate value theorem there exists
    $c_1\in(t_1,a)$, $c_2\in(a,t_2)$, and $c_3\in(t_2,b)$ such that
    $f(c_i)=y$ for $i=1,2,3$. Since $f(t_2)\ne y$, we know that
    the set $\{t\in P:u\cdot\gamma(t)=y\}$ is not an interval,
    and if it is finite, has cardinality $>2$; this is a contradiction
    to convexity. Now suppose that $f(t_1)<f(t_2)$.
    If $t_1\ne x$ then the proof is identical,
    so we assume that $t_1=x$. It is always possible to find
    a $\delta>0$ such that $x-\delta<t_1<t_2<x+p-\delta$.
    Similarly, there exists $c_1\in(x-\delta,t_1)$ and
    $c_2<c_3\in(t_1,t_2)$ such that $f(c_i)=y$ for $i=1,2,3$.
    Let $c_4=c_1+p\in(t_2,x+p)$ and obtain $f(c_4)=y$.
    Since $f(t_2)\ne y$ we again have a contradiction.
\end{proof}

The sufficiency part of the lemma seems way more tricky to prove
analytically, especially the case when $\mu(\gamma,u)=\infty$.
Milnor's proof, on the other hand, made too much use of
geometrical methods despite his analytical definitions.
It is not entirely clear, for example, why it is always possible
to rotate a line about one of its points of intersection with
a polygon so that the number of intersections is not decreased.
Also, we have this particular case when a polygon in the shape of 凸
(lit. convex; however the shape itself is concave)
is intersected by a horizontal line: the set of points of intersection
either has cardinality 2 or is infinite, but when it is infinite
the set may still be unconnected. This case is not seen to
be handled in Milnor's proof and does not seem trivial otherwise.
I think, however, that the lemma is true but it still needs
an analytical proof for the sake of completeness.

\section{Proof}

Now that we have sanity-checked the equivalency of definitions,
we shall first present a lemma and then our main theorem.

\begin{lemma}
    \label{lem}
    Let $f:(a,b)\cup(b,c)\to\mathbb{R}^2$ be a continuous function
    which is not constant, and let $p\in\mathbb{R}^2$.
    Suppose that there exists $x_1,x_2\in(a,b)\cup(b,c)$ such that
    the points $p,f(x_1),f(x_2)$ are noncollinear.
    Then there exists $x_1'\in(a,b),x_2'\in(b,c)$ such that
    the points $p,f(x_1'),f(x_2')$ are noncollinear.
\end{lemma}

\begin{proof}
    Suppose that for all $x_1'\in(a,b)$ and $x_2'\in(b,c)$,
    the points $p,f(x_1'),f(x_2')$ are collinear.
    Then we have $x_1,x_2\in A$ where
    $A$ is either $(a,b)$ or $(b,c)$. Let $B$ be the other interval.
    Since $f$ is not constant, there exists $x_0\in B$
    such that $f(x_0)\ne p$. But $f(x_0)$ cannot be collinear
    with $p,f(x_1)$ and with $p,f(x_2)$ at the same time.
\end{proof}

\begin{theorem}
    A convex closed curve is simple iff
    it does not lie entirely on a line.
\end{theorem}

\begin{proof}
    \newcommand{\gb}{\bar{\gamma}}
    Suppose that $\gamma$ is a simple closed curve in $\mathbb{E}^2$
    which lies entirely on a line. Then there exists a simple closed
    curve $\gb$ in $\mathbb{E}^1$.
    Let $P=[x,x+p)$ be any period of $\gb$. Since $\gb$ is not constant,
    there exists $t\in(x,x+p)$ such that $\gb(t)\ne\gb(x)=\gb(x+p)$.
    By the intermediate value theorem there exists $c_1\in(x,t),
        c_2\in(t,x+p)$ such that $\gb(c_1)=\gb(c_2)=y$ for some
    $y$ between $\gb(t)$ and $\gb(x)$. But then $0<(c_2-c_1)/p<1$,
    a contradiction to simpleness.

    Now suppose that $\gamma$ is a convex closed curve in $\mathbb{E}^2$
    which is not simple and does not lie entirely on a line.
    Let $X=[x_1,x_1+p)$ be a period of $\gamma$ such that
    $\gamma(x_1)=\gamma(x_1')$ for some $x_1'\in(x_1,x_1+p)$.
    Then there exists $x_2,x_3\in(x_1,x_1')\cup(x_1',x_1+p)$
    such that the points $\gamma(x_1),\gamma(x_2),\gamma(x_3)$ are
    noncollinear. By Lemma \ref{lem} we may assume that
    $x_2\in(x_1,x_1')$ and $x_3\in(x_1',x_1+p)$.
    Now let $P$ be any period of $\gamma$. Then there exists
    $c_1,c_1',c_2,c_3\in P$ such that $c_1<c_3<c_1'$,
    that $\gamma(c_1)=\gamma(c_1')$, and that the points
    $p_1=\gamma(c_1),p_2=\gamma(c_2),p_3=\gamma(c_3)$ are noncollinear.
    Let $u\in\mathbb{S}^1$ be a vector perpendicular to
    $\overline{p_1p_2}$. Then we have
    $u\cdot\gamma(c_1)=u\cdot\gamma(c_1')=u\cdot\gamma(c_2)
        \ne u\cdot\gamma(c_3)$, a contradiction to convexity.
\end{proof}

\begin{thebibliography}{9}

    \bibitem{milnor}
    J. W. Milnor,
    \emph{On the Total Curvature of Knots},
    Ann. of Math.
    \textbf{52} (1950), no. 2, 248-257.

\end{thebibliography}

\end{document}