\documentclass{amsart}
\usepackage{amsmath,amssymb,amsthm}
\usepackage{xeCJK}
\setCJKmainfont{SimHei}

\newtheorem{proposition}{Proposition}[section]
\newtheorem{lemma}[proposition]{Lemma}
\newtheorem{theorem}[proposition]{Theorem}
\newtheorem{conjecture}[proposition]{Conjecture}

\theoremstyle{definition}
\newtheorem{definition}[proposition]{Definition}

\theoremstyle{remark}
\newtheorem*{remark}{Remark}

\title{When is a convex closed curve not simple?}
\author{Scallop Ye}

\begin{document}

\begin{abstract}
    A convex curve is a plane curve whose intersection with a
    line is either a set of at most two points or a line segment.
    With analytical definitions of convexity and simpleness,
    we show that a convex closed curve is simple
    if and only if it does not lie entirely on a line.
    We conjecture that the same result still holds when
    geometrical definitions are used instead.
\end{abstract}

\maketitle

\section{Definitions}

We first follow Milnor \cite{milnor} to define the concepts in a rigorous manner.

\begin{definition}[Closed curves]
    \label{def:closed_curves}
    A \emph{closed curve} in Euclidean $n$-space $\mathbb{E}^n$ is a
    continuous periodic surjection $\gamma:\mathbb{R}\to C\subset\mathbb{R}^n$
    which is not constant on any interval.
\end{definition}

One can show that a closed curve has a least positive period.
For convenience we redefine the notion of period for closed curves
as follows.

\begin{definition}[Periods]
    \label{def:periods}
    Let $\gamma$ be a closed curve of least positive period $p$,
    and let $x\in\mathbb{R}$. We call $p$ \emph{the period} of $\gamma$,
    and the interval $P=[x,x+p)$ \emph{a period} of $\gamma$.
    We say that a closed curve $\phi$ is \emph{of period} $p'$
    iff the period of $\phi$ is $p'$.
\end{definition}

In fact, every closed curve $\gamma$ of period $p$
can be factored into two functions
$\gamma^*:\mathbb{S}^1\to C$ and
$f:\mathbb{R}\to\mathbb{S}^1$ such that $\gamma=\gamma^*\circ f$,
where $\gamma^*$ is a continuous surjection which is not constant
on any arc and $f(t):=(\cos 2\pi t/p,\sin 2\pi t/p)$.
We may call $\gamma^*$ the \emph{underlying map} of $\gamma$
and fix this notation.

Note that the restriction of $f$ to every period of $\gamma$ is bijective. This fact allows us to translate between
the properties of the periodic function $\gamma$
(on individual periods) and of the underlying map $\gamma^*$
(on the circle).

\begin{definition}[Simpleness]
    \label{def:simpleness}
    Let $\gamma$ be a closed curve of period $p$.
    We say that $\gamma$ is \emph{simple}
    iff for all $t_1,t_2\in\mathbb{R}$, $\gamma(t_1)=\gamma(t_2)$
    only when $(t_1-t_2)/p$ is an integer.
\end{definition}

\begin{proposition}
    \label{pro:simple_equiv}
    Let $\gamma$ be a closed curve.
    Then the following are equivalent:
    \begin{enumerate}
        \item $\gamma$ is simple.
        \item $\gamma^*$ is a homeomorphism.
        \item For every period $P$ of $\gamma$, the restriction
              of $\gamma$ to $P$ is bijective.
        \item There exists a period $P$ of $\gamma$ such that
              the restriction of $\gamma$ to $P$ is bijective.
    \end{enumerate}
\end{proposition}

\begin{proof}
    The claim follows immediately from the definitions above.
\end{proof}

\begin{proposition}
    \label{pro:simple_suf_cond}
    Let $f:\mathbb{R}\to C\subset\mathbb{R}^n$
    be a continuous periodic surjection,
    and let $p>0$ be a period of $f$. Suppose that
    the restriction of $f$ to some interval $[x_0,x_0+p)$
    is bijective.
    Then $f$ is a simple closed curve of period $p$.
\end{proposition}

\begin{proof}
    The restriction being bijective implies that
    $f$ is not constant on any interval and that
    $p$ is the least positive period of $f$. Thus $f$ is a 
    closed curve of period $p$. Then by Proposition
    \ref{pro:simple_equiv} we know that $f$ is simple.
\end{proof}

\begin{definition}[Convexity]
    Let $\gamma$ be a closed curve in $\mathbb{E}^2$.
    We say that $\gamma$ is \emph{convex} iff
    for every line $L\subset\mathbb{R}^2$,
    there exists a period $P$ of $\gamma$ such that
    the set $\{t\in P:\gamma(t)\in L\}$ either
    has cardinality $\le2$ or is a nondegenerate interval.
\end{definition}

Here we see that an arc on the circle corresponds
to an interval on \emph{some} period.

\begin{proposition}
    A closed curve $\gamma$ in $\mathbb{E}^2$ is convex iff
    for every $u\in\mathbb{S}^1$ and $y\in\mathbb{R}$,
    there exists a period $P$ of $\gamma$ such that
    the set $\{t\in P:u\cdot\gamma(t)=y\}$ either
    has cardinality $\le2$ or is a nondegenerate interval.
\end{proposition}

\begin{proof}
    A normal vector $u\in\mathbb{S}^1$ and
    a distance $y\in\mathbb{R}$ from the origin specifies
    a line $L=\{x\in\mathbb{R}^2:u\cdot x=y\}$, and
    every line in $\mathbb{R}^2$ can be specified in this manner.
\end{proof}

We then prove a generalization of the necessity part
of \cite[Lemma 3.3]{milnor} in the hope that
our definitions are equivalent to Milnor's.

\begin{definition}
    Let $\gamma$ be a closed curve in $\mathbb{E}^n$,
    let $u\in\mathbb{S}^{n-1}$,
    and let $P$ be any period of $\gamma$.
    Define $\mu(\gamma,u)$ to be the cardinality of the set
    $\{t\in P:\text{the function }u\cdot\gamma\text{ attains a local maximum at }t\}$
    if it is finite, or $\infty$ otherwise.
\end{definition}

\begin{proposition}
    Let $\gamma$ be a closed curve in $\mathbb{E}^2$.
    If $\gamma$ is convex, then for every $u\in\mathbb{S}^1$
    either $\mu(\gamma,u)=1$ or $\mu(\gamma,u)=\infty$.
\end{proposition}

\begin{proof}
    Suppose that $2\le\mu(\gamma,u)<\infty$ for some $u\in\mathbb{S}^1$.
    Let $f:\mathbb{R}\to\mathbb{R}$ be the function $u\cdot\gamma$,
    let $P=[x,x+p)$ be any period of $\gamma$, and let $t_1<t_2\in P$
    such that $f$ attains local maxima at $t_1$ and $t_2$.
    Suppose that $f(t_1)\ge f(t_2)$.
    Since $f$ attains a finite number of local maxima in $P$,
    there exists $a,b,y$ with $t_1<a<t_2<b<x+p$ such that
    $f(a),f(b)<y<f(t_2)\le f(t_1)$.
    By the intermediate value theorem there exists
    $c_1\in(t_1,a)$, $c_2\in(a,t_2)$, and $c_3\in(t_2,b)$ such that
    $f(c_i)=y$ for $i=1,2,3$. Since $f(t_2)\ne y$, we know that
    the set $\{t\in P:u\cdot\gamma(t)=y\}$ is not an interval,
    and if it is finite, has cardinality $>2$; this is a contradiction
    to convexity. Now suppose that $f(t_1)<f(t_2)$.
    If $t_1\ne x$ then the proof is identical,
    so we assume that $t_1=x$. It is always possible to find
    a $\delta>0$ such that $x-\delta<t_1<t_2<x+p-\delta$.
    Similarly, there exists $c_1\in(x-\delta,t_1)$ and
    $c_2<c_3\in(t_1,t_2)$ such that $f(c_i)=y$ for $i=1,2,3$.
    Let $c_4=c_1+p\in(t_2,x+p)$ and obtain $f(c_4)=y$.
    Since $f(t_2)\ne y$ we again have a contradiction.
\end{proof}

The sufficiency part of the lemma seems way more tricky to prove
analytically, especially the case when $\mu(\gamma,u)=\infty$.
Milnor's proof, on the other hand, made too much use of
geometrical methods despite his analytical definitions.
It is not entirely clear, for example, why it is always possible
to rotate a line about one of its points of intersection with
a polygon so that the number of intersections is not decreased.
Also, we have this particular case when a polygon in the shape of 凸
(lit. convex; however the shape itself is concave)
is intersected by a horizontal line: the set of points of intersection
either has cardinality 2 or is infinite, but when it is infinite
the set may still be disconnected. This case is not seen to
be handled in Milnor's proof and does not seem trivial otherwise.
I think, however, that the lemma is true but still needs
an analytical proof for the sake of completeness.

\section{Analytical Proof}

Now that we have checked the equivalence of definitions,
we shall first present a lemma and then our main theorem.

\begin{lemma}
    \label{lem}
    Let $f:(a,b)\cup(b,c)\to\mathbb{R}^2$ be a continuous
    nonconstant function, and let $p\in\mathbb{R}^2$.
    Suppose that there exists $x_1,x_2\in(a,b)\cup(b,c)$ such that
    the points $p,f(x_1),f(x_2)$ are noncollinear.
    Then there exists $x_1'\in(a,b),x_2'\in(b,c)$ such that
    the points $p,f(x_1'),f(x_2')$ are noncollinear.
\end{lemma}

\begin{proof}
    Suppose that for all $x_1'\in(a,b)$ and $x_2'\in(b,c)$,
    the points $p,f(x_1'),f(x_2')$ are collinear.
    Then we have $x_1,x_2\in A$ where
    $A$ is either $(a,b)$ or $(b,c)$. Let $B$ be the other interval.
    Since $f$ is nonconstant, there exists $x_0\in B$
    such that $f(x_0)\ne p$. But $f(x_0)$ cannot be collinear
    with $p,f(x_1)$ and with $p,f(x_2)$ at the same time.
\end{proof}

\begin{theorem}
    \label{thm:main}
    A convex closed curve is simple iff
    it does not lie entirely on a line.
\end{theorem}

\begin{proof}
    The image of a simple closed curve is homeomorphic to
    $\mathbb{S}^1$ and thus cannot be one-dimensional.
    Let $\gamma$ be a convex closed curve in $\mathbb{E}^2$
    which is not simple and does not lie entirely on a line.
    Let $X=[x_1,x_1+p)$ be a period of $\gamma$ such that
    $\gamma(x_1)=\gamma(x_1')$ for some $x_1'\in(x_1,x_1+p)$.
    Then there exists $x_2,x_3\in(x_1,x_1')\cup(x_1',x_1+p)$
    such that the points $\gamma(x_1),\gamma(x_2),\gamma(x_3)$ are
    noncollinear. By Lemma \ref{lem} we may assume that
    $x_2\in(x_1,x_1')$ and $x_3\in(x_1',x_1+p)$.
    Now let $P$ be any period of $\gamma$. Then there exists
    $c_1,c_1',c_2,c_3\in P$ such that $\gamma(c_1)=\gamma(c_1')$,
    that $c_1<c_3<c_1'$, and that the points
    $p_1=\gamma(c_1),p_2=\gamma(c_2),p_3=\gamma(c_3)$ are noncollinear.
    Let $u\in\mathbb{S}^1$ be a vector perpendicular to
    $\overline{p_1p_2}$. Then we have
    $u\cdot\gamma(c_1)=u\cdot\gamma(c_1')=u\cdot\gamma(c_2)
        \ne u\cdot\gamma(c_3)$, a contradiction to convexity.
\end{proof}

\section{Geometrical Consideration}

\begin{definition}[Simple sets]
    Let $C\subset\mathbb{R}^n$.
    We say that $C$ is \emph{simple}
    iff $C$ is homeomorphic to
    the unit circle $\mathbb{S}^1$.
\end{definition}

\begin{definition}[Weakly convex sets]
    Let $C\subset\mathbb{R}^2$. We say that $C$ is
    \emph{weakly convex}, or simply \emph{convex},
    iff for every line $L\subset\mathbb{R}^2$, the set $C\cap L$
    either has cardinality $\le2$ or is a nondegenerate line segment.
\end{definition}

\begin{definition}[Parameterizations]
    Let $\gamma$ be a closed curve in $\mathbb{E}^n$,
    and let $C\subset\mathbb{R}^n$.
    We say that $\gamma$ is
    a \emph{parameterization} of $C$ iff the image of $\gamma$
    is $C$.
\end{definition}

\begin{proposition}
    \label{pro:simple_par}
    A simple subset of $\mathbb{R}^n$ has
    a simple parameterization.
\end{proposition}

\begin{proof}
    Let $C$ be a simple subset of $\mathbb{R}^n$.
    Then there exists a continuous bijection $f:\mathbb{S}^1\to C$.
    Let $g:\mathbb{R}\to\mathbb{S}^1$ be the function
    $g(t):=(\cos t,\sin t)$. We know that $g$ is continuous
    on $\mathbb{R}$ and periodic with period $2\pi$, and that
    the restriction of $g$ to any interval $[x,x+2\pi)$
    is bijective. Let $\gamma:\mathbb{R}\to C$ be the function
    $f\circ g$. Then $\gamma$ is continuous
    on $\mathbb{R}$ and periodic with period $2\pi$, and the
    restriction of $\gamma$ to any interval $[x,x+2\pi)$ is bijective. It follows from Proposition \ref{pro:simple_suf_cond}
    that $\gamma$ is a simple closed curve of image $C$.
\end{proof}

\begin{proposition}
    \label{pro:simple_convex}
    A simple closed curve of convex image is convex.
\end{proposition}

\begin{proof}
    Let $\gamma$ be a simple closed curve of convex image $C$.
    It follows from Proposition \ref{pro:simple_equiv}
    that $\gamma^*$ is a homeomorphism.
    Let $L\subset\mathbb{R}^2$ be any line.
    Since $C$ is convex, the set $C\cap L$
    either has cardinality $\le2$ or is a nondegenerate line segment.
    Thus the inverse image $(\gamma^*)^{-1}(C\cap L)$
    either has cardinality $\le2$ or is a nondegenerate arc.
    Recall that an arc on the circle corresponds to an interval
    on some period, and so we have $\gamma$ convex by definition.
\end{proof}

\begin{definition}[Irreducibility]
    Let $\gamma$ be a closed curve of image $C$.
    We say that $\gamma$ is \emph{irreducible} iff for every period
    $P$ and every nonempty interval $(a,b)\subset P$
    with $\gamma(a)=\gamma(b)$, we have
    $\gamma(P\backslash(a,b))\ne C$.
\end{definition}

Being irreducible means that the curve does not contain
``redundant loops'' which do not contribute to its image
and can thus be removed.
This sounds like a good property indeed,
so let us make some conjectures on it.
It is conceivable that
\ref{con:irr_exists} already has a proof somewhere,
and that \ref{con:irr_simple} has a simple proof.

\begin{conjecture}
    \label{con:irr_exists}
    The image of a closed curve has an irreducible parameterization.
\end{conjecture}

\begin{conjecture}
    \label{con:irr_simple}
    An irreducible closed curve of simple image is simple.
\end{conjecture}

\begin{conjecture}
    \label{con:irr_convex}
    An irreducible closed curve of convex image is convex.
\end{conjecture}

% FIXME: It is not entirely clear that $\gamma$ is convex
%        in the first case.
% \begin{proof}[Proof that
%         $\ref{con:irr_simple}\land\ref{con:main}
%             \implies\ref{con:irr_convex}$]
%     Let $\gamma$ be an irreducible closed curve of convex
%     image $C$. If $C$ is a subset of a line, then $\gamma$
%     is clearly convex. If $C$ is not a subset of a line,
%     then we have $C$ simple by Conjecture \ref{con:main},
%     $\gamma$ simple by Conjecture \ref{con:irr_simple},
%     and thus $\gamma$ convex by Proposition \ref{pro:simple_convex}.
% \end{proof}

We then present two equivalent conjectures.
Note that Conjecture \ref{con:main} is the geometrical analog to
our main theorem.

\begin{conjecture}
    \label{con:convex}
    Let $C$ be the image of a closed curve. If $C$ is convex,
    then $C$ has a convex parameterization.
\end{conjecture}

\begin{proof}[Proof that
    $\ref{con:irr_exists}\land\ref{con:irr_convex}
        \implies\ref{con:convex}$]
    This implication is immediate.
\end{proof}

\begin{proof}[Proof that $\ref{con:main}\implies\ref{con:convex}$]
    If $C$ is not simple, then by Conjecture
    \ref{con:main} we know that $C$ is a subset of a line,
    which clearly is a line segment and has a convex parameterization.
    If $C$ is simple, then by Proposition \ref{pro:simple_par}
    there exists a simple parameterization $\gamma$ of $C$,
    which is convex by Proposition \ref{pro:simple_convex}.
\end{proof}

\begin{conjecture}
    \label{con:main}
    Let $C$ be the image of a closed curve. Suppose that $C$ is convex.
    Then $C$ is simple iff $C$ is not a subset of a line.
\end{conjecture}

\begin{proof}[Proof that $\ref{con:convex}\implies\ref{con:main}$]
    Clearly, a subset of a line cannot be simple.
    Suppose that $C$ is not a subset of a line. By Conjecture
    \ref{con:convex} there exists a convex parameterization
    $\gamma$ of $C$. Then we have $\gamma$ simple by
    Theorem \ref{thm:main}, and thus $C$ simple by
    Proposition \ref{pro:simple_equiv}.
\end{proof}

\begin{thebibliography}{9}

    \bibitem{milnor}
    J. W. Milnor,
    \emph{On the Total Curvature of Knots},
    Ann. of Math.
    \textbf{52} (1950), no. 2, 248-257.

\end{thebibliography}

\end{document}