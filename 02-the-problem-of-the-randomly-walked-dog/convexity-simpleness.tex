\documentclass{amsart}
\usepackage{accents,amsmath,amssymb,amsthm,xeCJK}
\setCJKmainfont{SimHei}

\newtheorem{proposition}{Proposition}[section]
\newtheorem{lemma}[proposition]{Lemma}
\newtheorem{theorem}[proposition]{Theorem}
\newtheorem{conjecture}[proposition]{Conjecture}

\theoremstyle{definition}
\newtheorem{definition}[proposition]{Definition}

\theoremstyle{remark}
\newtheorem*{remark}{Remark}

\newcommand{\lring}[1]{\accentset{\circ}{#1}}
\newcommand{\ltilde}[1]{\accentset{\sim}{#1}}

\title{When is a convex closed curve not simple?}
\author{Scallop Ye}

\begin{document}

\begin{abstract}
    A convex curve is a plane curve whose intersection with a
    line is either a set of at most two points or a line segment.
    With analytical definitions of convexity and simpleness,
    we show that a convex closed curve is simple
    if and only if it does not lie on a line.
    We conjecture that the same result still holds when
    geometrical definitions are used instead.
\end{abstract}

\maketitle

\section{Definitions}

We first follow Milnor \cite{milnor} to
define the concepts in a rigorous manner.

\begin{definition}[Closed curves]
    A \emph{closed curve} in Euclidean $n$-space $\mathbb{E}^n$
    is a nonconstant continuous periodic function
    $\gamma:\mathbb{R}\to\mathbb{R}^n$. We say that a closed curve
    is \emph{planar} iff its image lies on a line or in a plane.
\end{definition}

Here we do not restrict a closed curve to be light\footnote{
    A continuous function $f:S_1\to S_2$ is called \emph{light}
    iff $f^{-1}(y)$ is totally disconnected for each $y\in S_2$
    (see \cite[Definition 13.1]{nadler}). A closed curve
    is light iff it is not constant on any interval.
}, considering that every closed curve may be
reparameterized to be light, as we will see in Section
\ref{sec:geometrical_consideration}.

One can show that a closed curve has a least positive period.
For convenience we redefine the notion of period for closed curves
as follows.

\begin{definition}[Periods]
    Let $\gamma$ be a closed curve of least positive period $p$,
    and let $x\in\mathbb{R}$. We call $p$ \emph{the period} of $\gamma$,
    and the interval $P:=[x,x+p)$ \emph{a period} of $\gamma$.
    We say that a closed curve $\gamma'$ is \emph{of period} $p'$
    iff the period of $\gamma'$ is $p'$.
\end{definition}

In fact, every closed curve $\gamma$ of period $p$
can be factored into a unique pair of functions
$(\lring{\gamma}:\mathbb{S}^1\to\mathbb{R}^n,
    \ltilde{\gamma}:\mathbb{R}\to\mathbb{S}^1)$
such that $\gamma=\lring{\gamma}\circ\ltilde{\gamma}$,
where $\lring{\gamma}$ is a nonconstant continuous function
and $\ltilde{\gamma}(t):=(\cos 2\pi t/p,\sin 2\pi t/p)$.
We may call $\lring{\gamma}$ the \emph{underlying map} of $\gamma$,
call $\ltilde{\gamma}$ the \emph{intermediate map} of $\gamma$,
and fix these notations.

Note that the restriction of $\ltilde{\gamma}$ to
every period of $\gamma$ is bijective.
This fact allows us to translate between
the properties of a closed curve $\gamma$
(on individual periods) and of its underlying map
$\lring{\gamma}$ (on the unit circle).

\begin{definition}[Simpleness]
    Let $\gamma$ be a closed curve of period $p$.
    We say that $\gamma$ is \emph{simple}
    iff for all $t_1,t_2\in\mathbb{R}$, $\gamma(t_1)=\gamma(t_2)$
    only when $(t_1-t_2)/p$ is an integer.
\end{definition}

\begin{proposition}
    \label{pro:simple_equiv}
    Let $\gamma$ be a closed curve.
    Then the following are equivalent:
    \begin{enumerate}
        \item $\gamma$ is simple.
        \item $\lring{\gamma}$ is a homeomorphism.
        \item For every period $P$ of $\gamma$, the restriction
              of $\gamma$ to $P$ is injective.
        \item There exists a period $P$ of $\gamma$ such that
              the restriction of $\gamma$ to $P$ is injective.
    \end{enumerate}
\end{proposition}

\begin{proof}
    The claim follows immediately from the definitions above.
\end{proof}

\begin{proposition}
    \label{pro:simple_suf_cond}
    Let $f:\mathbb{R}\to\mathbb{R}^n$
    be a continuous periodic function,
    and let $p>0$ be a period of $f$. Suppose that
    the restriction of $f$ to some interval $[x_0,x_0+p)$
    is injective.
    Then $f$ is a simple closed curve of period $p$.
\end{proposition}

\begin{proof}
    The restriction being injective implies that
    $f$ is nonconstant and that $p$ is the least positive
    period of $f$. Thus $f$ is a closed curve of period $p$.
    Then by Proposition \ref{pro:simple_equiv}
    we know that $f$ is simple.
\end{proof}

We now present a unified definition of convex closed curves in
all dimensions. Later we will show that the property of convexity
implies planarity and is preserved under a linear isomorphism
(thus consistent across all dimensions).

\begin{definition}[Convexity]
    Let $\gamma$ be a closed curve in $\mathbb{E}^n$.
    We say that $\gamma$ is \emph{convex} iff
    for every hyperplane $H\subset\mathbb{R}^n$,
    the inverse image $\lring{\gamma}^{-1}(H)$ either
    has cardinality $\le2$, is a nondegenerate arc,
    or is the entire unit circle.
\end{definition}

\begin{proposition}
    \label{pro:convex_equiv}
    A closed curve $\gamma$ in $\mathbb{E}^n$ is convex
    iff for every $u\in\mathbb{S}^{n-1}$ and $y\in\mathbb{R}$,
    there exists a period $P$ of $\gamma$ such that
    the set $\{t\in P:u\cdot\gamma(t)=y\}$ either
    has cardinality $\le2$ or is a nondegenerate interval.
\end{proposition}

\begin{proof}[Sketch of proof]
    A normal vector $u\in\mathbb{S}^{n-1}$ and
    an offset $y\in\mathbb{R}$ from the origin specify
    a hyperplane $H:=\{x\in\mathbb{R}^n:u\cdot x=y\}$, and
    every hyperplane in $\mathbb{R}^n$ can be specified in this manner.
    It only remains to notice that
    an interval within \emph{some} period of the closed curve $\gamma$
    specifies an arc on the unit circle or the entire unit circle
    (the domain of the underlying map $\lring{\gamma}$),
    and every arc on the unit circle and the entire unit circle
    can be specified in this manner.
\end{proof}

\begin{proposition}
    \label{pro:convex_planar}
    A convex closed curve is planar.
\end{proposition}

\begin{proof}
    Let $\gamma$ be a closed curve in $\mathbb{E}^n$ ($n\ge 3$) which
    is not planar, and let $P$ be any period of $\gamma$.
    Then there exists $c_1<c_2<c_3<c_4\in P$ such that the points
    $p_1:=\gamma(c_1),p_2:=\gamma(c_2),
        p_3:=\gamma(c_3),p_4:=\gamma(c_4)$
    are noncoplanar. Let $H\subset\mathbb{R}^n$ be a hyperplane
    such that $p_1,p_3,p_4\in H$ but $p_2\notin H$, and let
    $u\in\mathbb{S}^{n-1}$ be a vector normal to $H$. Since
    $u\cdot\gamma(c_1)=u\cdot\gamma(c_3)=u\cdot\gamma(c_4)
        \ne u\cdot\gamma(c_2)$,
    the set $\{t\in P:u\cdot\gamma(t)=y\}$ is not an interval,
    and if it is finite, has cardinality $>2$. Hence $\gamma$
    is not convex by Proposition \ref{pro:convex_equiv}.
\end{proof}

\begin{proposition}
    Let $\gamma$ be a closed curve in $\mathbb{E}^2$ of image $C$,
    let $\gamma'$ be a planar closed curve in $\mathbb{E}^n$
    of image $C'$ where $n\ge2$, and
    let $P\subset\mathbb{R}^n$ be a plane such that $C'\subset P$.
    Suppose that there exists a linear isomorphism
    $T:\mathbb{R}^2\to P$ such that $\gamma'=T\circ\gamma$.
    Then $\gamma$ is convex iff $\gamma'$ is convex.
\end{proposition}

\begin{proof}[Sketch of proof]
    The intersection of the plane $P$ and
    a hyperplane $H\subset\mathbb{R}^n$ is either
    the empty set, a line, or the entire plane $P$ (when $n>2$);
    conversely, every line in $P$ and the entire plane $P$
    (when $n>2$) can be written as such an intersection.
    It only remains to notice that both $T$ and $T^{-1}$
    transforms lines into lines and the entire plane into
    the other plane, in a one-to-one manner.
\end{proof}

We then prove a generalization of the necessity part
of \cite[Lemma 3.3]{milnor} in the hope that
our definitions are equivalent to Milnor's.

\begin{definition}
    Let $\gamma$ be a closed curve in $\mathbb{E}^n$,
    let $u\in\mathbb{S}^{n-1}$,
    and let $P$ be any period of $\gamma$.
    Define $\mu(\gamma,u)$ to be the cardinality of the set
    $\{t\in P:\text{the function }u\cdot\gamma
        \text{ attains a local maximum at }t\}$
    if it is finite, or $\infty$ otherwise.
\end{definition}

\begin{proposition}
    Let $\gamma$ be a closed curve in $\mathbb{E}^n$.
    If $\gamma$ is convex, then for every $u\in\mathbb{S}^{n-1}$
    either $\mu(\gamma,u)=1$ or $\mu(\gamma,u)=\infty$.
\end{proposition}

\begin{proof}
    Suppose that $2\le\mu(\gamma,u)<\infty$
    for some $u\in\mathbb{S}^{n-1}$.
    Let $f:\mathbb{R}\to\mathbb{R}$ be the function $u\cdot\gamma$,
    let $P:=[x,x+p)$ be any period of $\gamma$, and let $t_1<t_2\in P$
    such that $f$ attains local maxima at $t_1$ and $t_2$.
    Suppose that $f(t_1)\ge f(t_2)$.
    Since $f$ attains a finite number of local maxima in $P$,
    there exists $a,b,y$ with $t_1<a<t_2<b<x+p$ such that
    $f(a),f(b)<y<f(t_2)\le f(t_1)$.
    By the intermediate value theorem there exists
    $c_1\in(t_1,a)$, $c_2\in(a,t_2)$, and $c_3\in(t_2,b)$ such that
    $f(c_i)=y$ for $i=1,2,3$. Since $f(t_2)\ne y$, we know that
    $\gamma$ is not convex, a contradiction.
    Now suppose that $f(t_1)<f(t_2)$.
    If $t_1\ne x$ then the proof is identical,
    so we assume that $t_1=x$. Take $\delta:=(x+p-t_2)/2$.
    Similarly, there exists $c_1,c_2,c_3,y$ with
    $x-\delta<c_1<t_1=x<c_2<c_3<t_2<x+p-\delta$
    and $y<f(t_1)<f(t_2)$ such that
    $f(c_i)=y$ for $i=1,2,3$.
    Let $c_4:=c_1+p$ and obtain $f(c_4)=y$ with $c_4\in(t_2,x+p)$.
    Since $f(t_2)\ne y$, we again have a contradiction.
\end{proof}

The sufficiency part of the lemma seems way more tricky to prove
analytically, especially the case when $\mu(\gamma,u)=\infty$.
Milnor's proof, on the other hand, made too much use of
geometrical methods despite his analytical definitions.
It is not entirely clear, for example, why it is always possible
to rotate a line about one of its points of intersection with
a polygon so that the number of intersections is not decreased.
Also, we have this particular case when a polygon in the shape of 凸
(lit. convex; however the shape itself is concave)
is intersected by a horizontal line: the set of points of intersection
either has cardinality 2 or is infinite, but when infinite
the set may still be disconnected. This case is not seen to
be handled in Milnor's proof and does not seem trivial otherwise.
I think, however, that the lemma is true but still needs
an analytical proof for the sake of completeness.

\section{Analytical Proof}

Now that we have checked the equivalence of definitions,
we shall first present a lemma and then our main theorem.
Note that it suffices to give proofs for the two-dimensional case
as a convex closed curve is planar.

\begin{lemma}
    \label{lem}
    Let $f:(a,b)\cup(b,c)\to\mathbb{R}^2$ be a
    nonconstant function, and let $p\in\mathbb{R}^2$.
    Suppose that there exists $x_1,x_2\in(a,b)\cup(b,c)$ such that
    the points $p,f(x_1),f(x_2)$ are noncollinear.
    Then there exists $x_1'\in(a,b)$ and $x_2'\in(b,c)$ such that
    the points $p,f(x_1'),f(x_2')$ are noncollinear.
\end{lemma}

\begin{proof}
    Suppose that for all $x_1'\in(a,b)$ and $x_2'\in(b,c)$,
    the points $p,f(x_1'),f(x_2')$ are collinear.
    Then we have $x_1,x_2\in A$ where
    $A$ is either $(a,b)$ or $(b,c)$. Let $B$ be the other interval.
    Since $f$ is nonconstant, there exists $x_0\in B$
    such that $f(x_0)\ne p$. But $f(x_0)$ cannot be collinear
    with $p,f(x_1)$ and with $p,f(x_2)$ at the same time.
\end{proof}

\begin{theorem}
    \label{thm:main}
    A convex closed curve is simple iff it does not lie on a line.
\end{theorem}

\begin{proof}
    The image of a simple closed curve is homeomorphic to
    $\mathbb{S}^1$ and thus cannot be one-dimensional.
    Let $\gamma$ be a convex closed curve in $\mathbb{E}^2$
    which is not simple and does not lie on a line.
    Let $X:=[x_1,x_1+p)$ be a period of $\gamma$ such that
    $\gamma(x_1)=\gamma(x_1')$ for some $x_1'\in(x_1,x_1+p)$.
    Then there exists $x_2,x_3\in(x_1,x_1')\cup(x_1',x_1+p)$
    such that the points $\gamma(x_1),\gamma(x_2),\gamma(x_3)$ are
    noncollinear. By Lemma \ref{lem} we may assume that
    $x_2\in(x_1,x_1')$ and $x_3\in(x_1',x_1+p)$.
    Now let $P$ be any period of $\gamma$. Then there exists
    $c_1,c_2,c_1',c_3\in P$ such that $\gamma(c_1)=\gamma(c_1')$,
    that $c_1<c_2<c_1'$, and that the points
    $p_1:=\gamma(c_1),p_2:=\gamma(c_2),p_3:=\gamma(c_3)$ are noncollinear.
    Let $u\in\mathbb{S}^1$ be a vector perpendicular to
    $\overline{p_1p_3}$. Then we have
    $u\cdot\gamma(c_1)=u\cdot\gamma(c_1')=u\cdot\gamma(c_3)
        \ne u\cdot\gamma(c_2)$, a contradiction to convexity.
\end{proof}

\section{Geometrical Consideration}
\label{sec:geometrical_consideration}

\begin{definition}[Parameterizations]
    Let $\gamma$ be a closed curve in $\mathbb{E}^n$,
    and let $C\subset\mathbb{R}^n$.
    We say that $\gamma$ is a \emph{parameterization}
    of $C$ iff the image of $\gamma$ is $C$.
    Two closed curves are said to be
    \emph{reparameterizations} of each other
    iff their images equal.
\end{definition}

\begin{proposition}
    \label{pro:light_repar}
    Every closed curve has a light reparameterization.
\end{proposition}

\begin{proof}
    The claim follows from \cite[Corollary 13.4]{nadler}
    by viewing maps $f:\mathbb{S}^1\to\mathbb{R}^n$
    as maps $f:[0,1]\to\mathbb{R}^n$ with $f(0)=f(1)$.
\end{proof}

\begin{definition}[Simple sets]
    Let $C\subset\mathbb{R}^n$.
    We say that $C$ is \emph{simple}
    iff $C$ is homeomorphic to
    the unit circle $\mathbb{S}^1$.
\end{definition}

\begin{proposition}
    \label{pro:simple_par}
    A simple subset of $\mathbb{R}^n$ has
    a simple parameterization.
\end{proposition}

\begin{proof}
    Let $C$ be a simple subset of $\mathbb{R}^n$.
    Then there exists a continuous injection
    $f:\mathbb{S}^1\to\mathbb{R}^n$ of image $C$.
    Let $g:\mathbb{R}\to\mathbb{S}^1$ be the function
    $g(t):=(\cos t,\sin t)$. We know that $g$ is continuous
    on $\mathbb{R}$ and periodic with period $2\pi$, and that
    the restriction $g|_{[0,2\pi)}$ is injective.
    Let $\gamma:\mathbb{R}\to\mathbb{R}^n$ be the function
    $f\circ g$. Then $\gamma$ is continuous
    on $\mathbb{R}$ and periodic with period $2\pi$, and the
    restriction $\gamma|_{[0,2\pi)}$ is injective.
    It follows from Proposition \ref{pro:simple_suf_cond}
    that $\gamma$ is a simple closed curve of image $C$.
\end{proof}

\begin{definition}[Weakly convex sets]
    Let $C\subset\mathbb{R}^2$. We say that $C$ is \emph{weakly convex}
    iff for every line $L\subset\mathbb{R}^2$, the set $C\cap L$
    either has cardinality $\le2$ or is a nondegenerate line segment.
\end{definition}

\begin{proposition}
    \label{pro:simple_convex}
    A simple closed curve of weakly convex image is convex.
\end{proposition}

\begin{proof}
    Let $\gamma$ be a simple closed curve of weakly convex image $C$,
    and let $L\subset\mathbb{R}^2$ be any line.
    Since $C$ is weakly convex, the set $C\cap L$
    either has cardinality $\le2$ or is a nondegenerate line segment.
    Since $\lring{\gamma}$ is a homeomorphism by Proposition
    \ref{pro:simple_equiv}, the inverse image $\lring{\gamma}^{-1}(L)$
    either has cardinality $\le2$, is a nondegenerate arc, or is
    the entire unit circle. Thus $\gamma$ is convex by definition.
\end{proof}

\begin{definition}[Irreducibility]
    Let $\gamma$ be a closed curve of image $C$.
    We say that $\gamma$ is \emph{irreducible} iff for every period
    $P$ of $\gamma$ and every closed interval $[a,b]\subset P$
    with $\gamma(a)=\gamma(b)$, we have
    $\gamma([a,b])\ne C$.
\end{definition}

Being irreducible means that the curve does not contain
``redundant loops'' which do not contribute to its image
and can thus be removed.
This sounds like a good property indeed,
so let us make some conjectures on it.
It is conceivable that
\ref{con:irr_exists} already has a proof somewhere,
and that \ref{con:irr_simple} and/or
\ref{con:irr_one_dimensional_convex} have simple proofs.

\begin{conjecture}
    \label{con:irr_exists}
    Every closed curve has an irreducible reparameterization.
\end{conjecture}

\begin{conjecture}
    \label{con:irr_simple}
    An irreducible closed curve of simple image is simple.
\end{conjecture}

\begin{conjecture}
    \label{con:irr_one_dimensional_convex}
    An irreducible closed curve that lies on a line is convex.
\end{conjecture}

\begin{conjecture}
    \label{con:irr_convex}
    An irreducible closed curve of weakly convex image is convex.
\end{conjecture}

\begin{proof}[Proof that
        $\ref{con:irr_simple}
            \land\ref{con:irr_one_dimensional_convex}
            \land\ref{con:main}
            \implies\ref{con:irr_convex}$]
    Let $\gamma$ be an irreducible closed curve of weakly convex
    image $C$. If $C$ is a subset of a line, then $\gamma$
    is convex by Conjecture \ref{con:irr_one_dimensional_convex}.
    If $C$ is not a subset of a line,
    then we have $C$ simple by Conjecture \ref{con:main},
    $\gamma$ simple by Conjecture \ref{con:irr_simple},
    and thus $\gamma$ convex by Proposition \ref{pro:simple_convex}.
\end{proof}

We then present two equivalent conjectures.
Note that Conjecture \ref{con:main} is the geometrical analog to
our main theorem.

\begin{conjecture}
    \label{con:convex}
    Let $C$ be the image of a closed curve. If $C$ is weakly convex,
    then $C$ has a convex parameterization.
\end{conjecture}

\begin{proof}[Proof that
        $\ref{con:irr_exists}
            \land\ref{con:irr_convex}
            \implies\ref{con:convex}$]
    This implication is immediate.
\end{proof}

\begin{proof}[Proof that $\ref{con:main}\implies\ref{con:convex}$]
    If $C$ is not simple, then by Conjecture
    \ref{con:main} we know that $C$ is a subset of a line,
    which clearly is a line segment and has a convex parameterization.
    If $C$ is simple, then by Proposition \ref{pro:simple_par}
    there exists a simple parameterization $\gamma$ of $C$,
    which is convex by Proposition \ref{pro:simple_convex}.
\end{proof}

\begin{conjecture}
    \label{con:main}
    Let $C$ be the image of a closed curve.
    Suppose that $C$ is weakly convex.
    Then $C$ is simple iff $C$ is not a subset of a line.
\end{conjecture}

\begin{proof}[Proof that $\ref{con:convex}\implies\ref{con:main}$]
    Clearly, a subset of a line cannot be simple.
    Suppose that $C$ is not a subset of a line. By Conjecture
    \ref{con:convex} there exists a convex parameterization
    $\gamma$ of $C$. Then we have $\gamma$ simple by
    Theorem \ref{thm:main}, and thus $C$ simple by
    Proposition \ref{pro:simple_equiv}.
\end{proof}

\begin{thebibliography}{9}

    \bibitem{milnor}
    J. W. Milnor,
    \emph{On the Total Curvature of Knots},
    Ann. of Math.
    \textbf{52} (1950), no. 2, 248-257.

    \bibitem{nadler}
    S. B. Nadler, Jr.,
    \emph{Continuum Theory: An Introduction},
    CRC Press, Boca Raton, 1992.

\end{thebibliography}

\end{document}